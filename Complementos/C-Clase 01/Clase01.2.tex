\documentclass[a4paper,10pt]{article}
\usepackage[utf8]{inputenc}
\usepackage[T1]{fontenc}
\usepackage[spanish]{babel}
\usepackage{verbatim}
\usepackage{verbments}
\usepackage{marvosym}
\usepackage[marvosym]{tikzsymbols}
\usepackage{amsmath}
\usepackage{amssymb}
\usepackage{hyperref}
\hypersetup{
    bookmarks=true,         % show bookmarks bar?
    unicode=false,          % non-Latin characters in Acrobat’s bookmarks
    pdftoolbar=true,        % show Acrobat’s toolbar?
    pdfmenubar=true,        % show Acrobat’s menu?
    pdffitwindow=false,     % window fit to page when opened
    pdfstartview={FitH},    % fits the width of the page to the window
    pdftitle={My title},    % title
    pdfauthor={Author},     % author
    pdfsubject={Subject},   % subject of the document
    pdfcreator={Creator},   % creator of the document
    pdfproducer={Producer}, % producer of the document
    pdfkeywords={keyword1, key2, key3}, % list of keywords
    pdfnewwindow=true,      % links in new PDF window
    colorlinks=false,       % false: boxed links; true: colored links
    linkcolor=red,          % color of internal links (change box color with linkbordercolor)
    citecolor=green,        % color of links to bibliography
    filecolor=magenta,      % color of file links
    urlcolor=cyan           % color of external links
}
\usepackage[lmargin=2cm, rmargin=1cm, tmargin=2cm, bmargin=2cm]{geometry}
\definecolor{fondo1}{rgb}{0.96, 0.96, 0.96}
\definecolor{fondo2}{rgb}{0.3, 0.6, 0.5}
\definecolor{links}{rgb}{0.2, 0.2, 1.0}
\title{Adiciones a la Clase 01}
\author{HeNeos}
\begin{document}
\fvset{frame=bottomline, framerule=0.02cm}
\plset{language=latex,texcl=true, listingnamefont=\sffamily\bfseries\color{white},captionbgcolor=fondo2, bgcolor=fondo1,listingname=\textbf{Iniciando}, captionfont=\sffamily\color{white}}
	\maketitle
En la clase anterior vimos como es posible manejar colores, imágenes, tamaño y fuentes en \LaTeX, pero aún no hemos visto como podemos hacer modificaciones al texto $\ldots$ específicamente nos referimos a algunas instrucciones en \LaTeX\, que nos permiten modificar la estructura que vamos redactando sin necesidad de implicar un cambio global en el documento.
\section{Configuración de márgenes, interlineado e indentación}
En el preámbulo del documento agregamos lo siguiente:
\begin{pyglist}[language=latex,caption={Márgenes del documento},style=pastie]
%--------Preambulo--------------------------------------------------------------------
\documentclass[a4paper, 10pt]{article}
%
%
%-------------Configuracion de margen-------------
\usepackage[lmargin=3cm, rmargin=2.5cm, tmargin=3cm, bmargin=2.5cm]{geometry}
%lmargin = margen izquierdo
%rmargin = margen derecho
%tmargin = margen superior
%bmargin = margen inferior
%-------------------------------------------------
%
%
%-------------Configuracion del interlineado------
\linespread{n}
%Donde $n$ es un numero adimensional indicando el espacio interlineal
%-------------------------------------------------
%
%
%-------------Indentacion-------------------------
%Al usar la sentencia noindent al inicio de un parrafo se elimina su indentacion por defecto
\noindent
%-------------------------------------------------
%
%
\author{HeNeos}
%--------Fin del preambulo------------------------------------------------------------
\begin{document}
...
...
...
\end{document}
\end{pyglist}
\section{Control de texto}
\subsection{hspace y vspace}
Dos de los comandos más útiles a la hora de editar texto son \texttt{$\backslash$hspace\{$n$\}} y \texttt{$\backslash$vspace\{$n$\}}, el primero nos inserta un espacio horizontal de tamaño $n$, mientras que el segundo un tamaño vertical de tamaño $n$. Note que, $n$ representa una longitud, por lo que debe ser un número con su respectiva medida; es decir, \texttt{$\backslash$hspace\{5\}} es incorrecto, mientras que \texttt{$\backslash$vspace\{5pt\}} es correcto porque indicamos que la longitud es de tamaño 5pt. No hemos hablado demasiado de la unidad ``pt'', en \LaTeX\, es una unidad que equivale a $\frac{1}{72.27}\,$inch; como es una unidad de medida podemos reemplazarla también por cm, m, ft, etc. Adicionalmente, se puede usar una versión \textit{forzada} de los mismos añadiendo un asterisco. \href{https://tex.stackexchange.com/questions/89082/hspace-vs-hspace}{\textcolor{links}{\underline{\textbf{Revisar este link.}}}}
\subsection{hfill y vfill}
Similares a \texttt{hpsace} y \texttt{vspace}; con la diferencia que estos rellenan un espacio hasta el margen:\\
Escribiendo una línea con \hfill \texttt{hfill}.\\
Escribiendo una línea con \vfill \texttt{vfill}.
\newpage
\section{Alinear texto}
Para alinear un texto tenemos tres opciones: por la izquierda, centrado y por la derecha.
\begin{pyglist}[language=latex,caption={Justificacion del texto},style=pastie]
%--------Preambulo--------------------------------------------------------------------
\documentclass[a4paper, 10pt]{article}
%-------------Justificacion de texto-------------
\begin{flushleft}
Este parrafo esta justificado por la izquierda
Este parrafo esta justificado por la izquierda
Este parrafo esta justificado por la izquierda
Este parrafo esta justificado por la izquierda


Este parrafo esta justificado por la izquierda

Este parrafo esta justificado por la izquierda
\end{flushleft}
\begin{center}
Este parrafo esta centrado
Este parrafo esta centrado
Este parrafo esta centrado

Este parrafo esta centrado

Este parrafo esta centrado
\end{center}
\begin{flushright}
Este parrafo esta justificado por la derecha
Este parrafo esta justificado por la derecha
Este parrafo esta justificado por la derecha
Este parrafo esta justificado por la derecha

Este parrafo esta justificado por la derecha

Este parrafo esta justificado por la derecha
\end{flushright}
%--------Fin del preambulo------------------------------------------------------------
\begin{document}
...
...
...
\end{document}
\end{pyglist}
El código anterior nos da como resultado:
\begin{flushleft}
Este parrafo esta justificado por la izquierda
Este parrafo esta justificado por la izquierda
Este parrafo esta justificado por la izquierda
Este parrafo esta justificado por la izquierda


Este parrafo esta justificado por la izquierda

Este parrafo esta justificado por la izquierda
\end{flushleft}
\begin{center}
Este parrafo esta centrado
Este parrafo esta centrado
Este parrafo esta centrado

Este parrafo esta centrado

Este parrafo esta centrado
\end{center}
\begin{flushright}
Este parrafo esta justificado por la derecha
Este parrafo esta justificado por la derecha
Este parrafo esta justificado por la derecha
Este parrafo esta justificado por la derecha

Este parrafo esta justificado por la derecha

Este parrafo esta justificado por la derecha
\end{flushright}
\section{Listas y enumeraciones}
Muchas veces, queremos enumerar algo o simplemente listarlo. Para ello tenemos disponible dos opciones:
\begin{pyglist}[language=latex,caption={Márgenes del documento},style=pastie]
%--------Preambulo--------------------------------------------------------------------
\documentclass[a4paper, 10pt]{article}
%-------------Listar elementos-------------
\begin{itemize}
\item Esto
\item es
\item una
\item lista sin enumerar
\end{itemize}

\begin{enumerate}
\item Esto
\item es
\item una
\item lista
\item enumerada
\begin{itemize}
\item una lista
\item no enumerada
\item dentro de una enumerada
\begin{enumerate}
\item y ahora vuelta a enumerar
\end{enumerate}
\item sin enumerar
\end{itemize}
\item vuelto a enumerar
\end{enumerate}
%------------------------------------------
\end{pyglist}
Lo que da como resultado:
\begin{itemize}
\item Esto
\item es
\item una
\item lista sin enumerar
\end{itemize}

\begin{enumerate}
\item Esto
\item es
\item una
\item lista
\item enumerada
\begin{itemize}
\item una lista
\item no enumerada
\item dentro de una enumerada
\begin{enumerate}
\item y ahora vuelta a enumerar
\end{enumerate}
\item sin enumerar
\end{itemize}
\item vuelto a enumerar
\end{enumerate}
\section{Complementos random}
\subsection{Agregar un índice}
En la sesión anterior vimos como es posible crear una portada simple de forma rápida; ahora veremos como es posible crear una tabla de contenidos o índice. Para ello usaremos el comando \texttt{$\backslash$tableofcontents} dentro del cuerpo del documento, este puede estar en cualquier ubicación, aunque lo normal es situarlo justo después de la carátula, por lo que iría después de nuestro \texttt{$\backslash$maketitle}.
\subsection{Página nueva}
Para iniciar una página nueva en nuestro documento podemos usar las instrucciones \texttt{$\backslash$newpage} y \texttt{$\backslash$clearpage}.
\subsection{Alargar página}
Podemos ``alargar'' un tamaño pequeño nuestra página con la instrucción \texttt{$\backslash$enlargethispage*\{n\}}, aquí $n$ representa una longitud pequeña, por lo que debería estar en mm o pt.
\end{document}
